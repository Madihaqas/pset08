\documentclass[a4paper]{exam}


\usepackage{amsfonts,amsmath,amsthm}
\usepackage[a4paper]{geometry}
\usepackage{hyperref}
\usepackage{xcolor}

\theoremstyle{definition}
\newtheorem{definition}{Definition}

\newcommand\Z{\ensuremath{\mathbb{Z}}}
\newcommand\R{\ensuremath{\mathbb{R}}}

\title{Problem Set 08: Sets And Functions}
\author{CS/MATH 113 Discrete Mathematics}
\date{Spring 2024}

\boxedpoints

\printanswers

\begin{document}
\maketitle

\begin{questions}
\question 
  Prove: $\mathcal{P}(A) \subseteq \mathcal{P}(B)$ if and only if $A \subseteq B$.

  \begin{solution}
  
  If $\mathcal{P}(A) \subseteq \mathcal{P}(B)$ then $A \subseteq B$ and  If $A \subseteq B$ then $\mathcal{P}(A) \subseteq \mathcal{P}(B)$.\\
To prove the first part:\\
Assume $\mathcal{P}(A) \subseteq \mathcal{P}(B)$.\\
Taking any subset $C$ of $A$.\\
Since $C \subseteq \mathcal{P}(A)$, this implies $C \subseteq \mathcal{P}(B)$.\\
For any element $c$ in $C$, $c$ is a subset of $A$. Since $A \subseteq B$, this means $c \subseteq B$. Hence, $C \subseteq B$.\\
Since $C$ was chosen arbitrarily, this shows $A \subseteq B$.\\
To prove the second part:\\
Assume $A \subseteq B$.\\
Assuming a subset $C$ of $A$.\\
For any element $c$ in $C$, $c \in A$ and since $A \subseteq B$, $c \in B$.
This implies that $C \subseteq B$.\\
Therefore, $\mathcal{P}(A) \subseteq \mathcal{P}(B)$.\\
Hence proved
  \end{solution}


\question Show that if \(A \subseteq C\) and \(B \subseteq D\), then \(A \times B \subseteq C \times D\).

  \begin{solution}
    
     \(A \subseteq C\) = every element in $A$ is also in $C$.\\
     \(B \subseteq D\) = if an element is in $B$ then it must also be in $D$.\\
     Suppose $(a,b)$ $\in$ \(A \times B \)\\
     Then $a \in A$ and $b \in B$\\
     If $a \in A$ then $a \in C$\\
      If $b \in B$ then $b \in D$\\
      This shows that $(a,b)$ $\in$ \(C \times D \)\\
      Then,  \(A \times B \subseteq C \times D\).
  \end{solution}
  
\question Prove or disprove the following statements for all sets $A, B,$ and $C$:
  \begin{parts}
  \part $ A \times (B \cup C) = (A \times B) \cup (A \times C) $
    \begin{solution}
    
      $ A \times (B \cup C)$ = ${(a,b) | a \in A \land b \in(B \cup C) }$\\
      ${(a,b) | a \in A \land ( b \in B )\lor (b \in C) }$\\
      Using distributive law.\\
       ${ (a \in A \land ( b \in B )) \lor (a \in A\land (b \in C)) }$\\
      $(A \times B) \lor (A \times C)$\\
      $(A \times B) \cup (A \times C)$\\
      Hence proved
    \end{solution}

  \part $A \times (B \cap C) = (A \times B) \cap (A \times C) $
    \begin{solution}
    
      $ A \times (B \cap C)$ = ${(a,b) | a \in A \land b \in(B \cap C) }$\\
      ${(a,b) | a \in A \land ( b \in B )\land (b \in C) }$\\
      Using distributive law.\\
       ${ (a \in A \land ( b \in B )) \land (a \in A\land (b \in C)) }$\\
      $(A \times B) \land (A \times C)$\\
      $(A \times B) \cap (A \times C)$\\
      Hence proved
    \end{solution}
  \end{parts}
  
\question If $A, B,$ and $C$ are sets such that $A \subseteq B$ and $B \subseteq C$, show that $A \subseteq C$.

  \begin{solution}
    
    $\forall x(x \in A \implies x\in B)$ $\land$  $\forall x(x \in B \implies x\in C)$\\
    This shows that if $x$ belongs to $A$ then $x$ must also belongs to $B$ and if $x$ belongs to $B$ then $x$ must also belong to $C$.\\
    Therefore, if $x$ $\in$ $A$ then $x$ $\in$ $C$.\\
    $A \subseteq C$\\
    Hence proved
  \end{solution}

\question Show that if $A$ and $B$ are finite sets, then  \( A \cup B \) is a finite set.

  \begin{solution}

    suppose $A=\{a_1,a_2,...,a_n\}$ and $B= \{b_1,b_2,...,b_n\}$ are two finite sets.\\
    Then their union would be:\\
     \( A \cup B \) = $\{a_1,b_1,a_2,b_2,...a_n,b_n\}$\\
     Which is also finite
  \end{solution}
  
\question
  \begin{parts}
  \part Prove that the function $f: \Z\to\Z$, defined as \(f(n) = n^2\) is neither injective nor surjective.
    \begin{solution}
      
      The function is not injective:\\
      Proof using counterexample\\
      \(f(n) = n^2\)\\
      $f(2)=f(-2)$ but 2 $\not =$ -2.\\
       The function is not surjective:\\
      no such number exist whose square is a negative number, this does not cover the domain of all integers hence it is not surjective.
      
    \end{solution}

  \part Can you make this function injective without doing any changes to the original function? (Hint: try redefining its domain and/or co-domain.)
    \begin{solution}
    
      Changing the domain and codomain from $f: \Z\to\Z$ to $f: \Z^+\to\Z^+$ would make this function injective.
      
    \end{solution}
  \end{parts}

  
\question Prove that the function $f: \Z\to\Z$ defined as \(f(n) = n+1\), is invertible.

  \begin{solution}
    
    A function is invertible if it has an inverse. To prove that the inverse exist, we need to show that the function is both injective and surjective.\\
    To show that the function is one-to-one:\\
    \(f(n) = n+1\) taking a and b where a $\not =$ b\\
    then \(f(a) = a+1\) $\not =$ \(f(b) = b+1\)
    To show that the function is onto:\\
    \(f(n) = n+1\) we need to demonstrate that for every integer $y$ in the codomain, there exists an integer $x$ in the domain such that $f(x)=y$.\\
    to find an integer $x$ in the codomain such that $f(x)=y$.\\
    $x$ +1 = $y$\\
    $x$ = $y$ -1\\
    This shows that every element in codomain has at least one match in the domain, so the function is onto.\\
    We have proved that the function has one to one correspondence so it is invertible.
     
  \end{solution}
  
\question Give an explicit formula for a function from \Z to $\Z^+$ that is:
  \begin{parts}
  \part one-to-one, but not onto,
    \begin{solution}
     
     $f(n)=n^2+1$ 
    \end{solution}

  \part onto, but not one-to-one,
    \begin{solution}
      
     $f(n)=n^2$ 
    \end{solution}

  \part one-to-one and onto, and
    \begin{solution}
      
      $f(n)=n+1$ 
      
    \end{solution}

  \part neither one-to-one nor onto.
    \begin{solution}
      
      $-n^2$
    \end{solution}
  \end{parts}

\question A function \(f: I \rightarrow \R\) is strictly increasing on an interval \(I\) if for all \(x_1, x_2 \in I\) with \(x_1 < x_2\), it holds that \(f(x_1) < f(x_2)\). Here $I$ represents the interval on which the function is strictly increasing.
  
  Given this definition, prove that a strictly increasing function is always injective over the interval, I.

  \begin{solution}
    
  By definition, if $a < b$ then $f(a) < f(b)$\\
  and if $a > b$ then $f(a) > f(b)$, \\
  Thus if $a \not = b$ then  $f(a) \not = f(b)$ hence the functions will always be injective.
    
    
  \end{solution}
\end{questions}
\end{document}
%%% Local Variables:
%%% mode: latex
%%% TeX-master: t
%%% End:
